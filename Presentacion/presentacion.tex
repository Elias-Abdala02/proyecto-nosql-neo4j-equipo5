\documentclass[aspectratio=169]{beamer}
\usepackage[spanish]{babel}
\usepackage[utf8]{inputenc}
\usepackage[T1]{fontenc}
\usepackage{hyperref}

\usetheme{Madrid}
\usecolortheme{dolphin}

\title{Proyecto Extra\\Aplicación Web con FastAPI y Neo4J}
\author{Equipo 5}
\date{Agosto--Diciembre 2025}
\institute{Universidad Autónoma de Yucatán\\Modelos de Datos LCC}

\begin{document}

\begin{frame}
  \titlepage
  \begin{center}
    \small Profesor: M. en C. Luis R. Basto Díaz
  \end{center}
\end{frame}

\begin{frame}{Agenda}
  \tableofcontents
\end{frame}

\section{Objetivo}

\begin{frame}{Meta del Proyecto Extra}
  \begin{itemize}
    \item Exponer las operaciones CRUD y el modelo de grafos en una aplicación web simple.
    \item Levantar API y Neo4J con Docker Compose para asegurar reproducibilidad.
    \item Ofrecer una UI ligera que permita: seed de datos, CRUD rápido y visualización interactiva del grafo.
  \end{itemize}
\end{frame}

\section{Arquitectura}

\begin{frame}{Componentes principales}
  \begin{itemize}
    \item \textbf{docker-compose.yml} con dos servicios:
    \begin{itemize}
      \item \textbf{neo4j} (v5.15): monta \texttt{neo4j-data/} y el CSV en \texttt{/import/shopping\_behavior.csv}. Credenciales \texttt{neo4j/test1234}.
      \item \textbf{app} (FastAPI): cliente oficial de Neo4J, expuesta en \texttt{http://localhost:8000}.
    \end{itemize}
    \item Red interna: \texttt{proyecto-extra-net}.
  \end{itemize}
\end{frame}

\begin{frame}{Estructura del repositorio}
  \begin{itemize}
    \item \texttt{proyecto-extra/docker-compose.yml}: orquesta API + Neo4J.
    \item \texttt{proyecto-extra/app/}:
    \begin{itemize}
      \item \texttt{main.py}: endpoints \textbf{/seed}, CRUD, grafo, health.
      \item \texttt{static/index.html}: UI HTML/JS con vis-network.
      \item Dockerfile y requirements para construir la imagen de la API.
    \end{itemize}
    \item \texttt{neo4j/}: scripts Cypher originales (montados en el contenedor).
  \end{itemize}
\end{frame}

\section{Dataset y carga}

\begin{frame}{Dataset y seed}
  \begin{itemize}
    \item Dataset: \texttt{data/shopping\_behavior.csv} (3{,}900 transacciones).
    \item Endpoint \textbf{/seed}:
    \begin{itemize}
      \item Crea constraints (Customer, Product, Category).
      \item Carga nodos y relaciones con \texttt{LOAD CSV}.
    \end{itemize}
    \item Acceso Neo4J: \texttt{bolt://localhost:7687}, usuario \texttt{neo4j}, password \texttt{test1234}.
  \end{itemize}
\end{frame}

\section{API}

\begin{frame}{Endpoints CRUD}
  \begin{itemize}
    \item CREATE: \texttt{/customers}, \texttt{/categories}, \texttt{/products}, \texttt{/purchases}, \texttt{/products/with-category}.
    \item READ: clientes >50, top productos, clientes por categoría, resumen por método de pago, clientes premium.
    \item UPDATE: edad, suscripción por ubicación, rating promedio, incremento de compras previas, actualización de producto.
    \item DELETE: cliente + relaciones, compras con rating bajo, productos sin compras, relación producto-categoría, clientes inactivos.
    \item Utilidades: \texttt{/seed}, \texttt{/health}, Swagger en \texttt{/docs}.
  \end{itemize}
\end{frame}

\begin{frame}{Endpoints de grafo}
  \begin{itemize}
    \item \texttt{/graph/options}: devuelve valores disponibles para categoría, producto o cliente (pobla los selects de la UI).
    \item \texttt{/graph/sample}: genera un subgrafo centrado en el nodo elegido, con profundidad y límite configurables; devuelve nodos y relaciones para pintar en la UI.
  \end{itemize}
\end{frame}

\section{UI Web}

\begin{frame}{UI (index.html)}
  \begin{itemize}
    \item Botones rápidos: \textbf{seed}, \textbf{health}, \textbf{top productos}.
    \item Formularios CRUD: crear cliente, actualizar edad, eliminar cliente.
    \item Visor de grafo (vis-network):
    \begin{itemize}
      \item Selecciona centro (categoría/producto/cliente) y valor desde listas dinámicas.
      \item Ajusta profundidad y límite; clic en nodos muestra propiedades.
      \item Física ajustada para separar nodos y reducir solapamiento de aristas.
    \end{itemize}
  \end{itemize}
\end{frame}

\section{Flujo de uso}

\begin{frame}{Paso a paso}
  \begin{enumerate}
    \item \texttt{docker compose up -d} (en \texttt{proyecto-extra/}).
    \item Ejecutar \textbf{/seed} desde la UI o vía HTTP si es la primera vez.
    \item Probar health y un READ (\texttt{/read/top-products} o el botón en la UI).
    \item CRUD rápido desde la UI (crear/actualizar/eliminar cliente).
    \item Visualizar el grafo: elegir centro/valor/profundidad y explorar nodos.
  \end{enumerate}
\end{frame}

\section{Conclusiones}

\begin{frame}{Conclusiones}
  \begin{itemize}
    \item La app web encapsula el modelo y CRUD en una interfaz sencilla.
    \item Docker Compose garantiza reproducibilidad: API + Neo4J listos con un comando.
    \item El visor de grafo permite explorar relaciones cliente–producto–categoría directamente en el navegador.
  \end{itemize}
\end{frame}

\begin{frame}
  \centering
  \Huge ¡Gracias!
\end{frame}

\end{document}
