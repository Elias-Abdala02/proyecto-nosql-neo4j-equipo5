\documentclass[aspectratio=169]{beamer}

\usepackage[spanish]{babel}
\usepackage{graphicx}
\usepackage{hyperref}
\usepackage{booktabs}

% Si quieres, puedes usar fontspec con XeLaTeX:
% \usepackage{fontspec}
% \setsansfont{Latin Modern Sans}

\usetheme{Madrid}
\usecolortheme{dolphin}

\title{Proyecto NoSQL -- Neo4J\\Análisis de comportamiento de compra}
\author{Equipo 5}
\date{Agosto--Diciembre 2025}
\institute{Universidad Autónoma de Yucatán\\Modelos de Datos LCC}

\begin{document}

\begin{frame}
  \titlepage
  \begin{center}
    \small Profesor: M. en C. Luis R. Basto Díaz
  \end{center}
\end{frame}

\begin{frame}{Agenda}
  \tableofcontents
\end{frame}

\section{Dataset}

\begin{frame}{Fuente y descripción}
  \begin{itemize}
    \item Kaggle: \textit{Shopping Behaviour and Product Ranking Dataset}.
    \item 3{,}900 transacciones con datos de clientes, productos, categorías, monto y rating.
    \item Archivo principal: \texttt{data/shopping\_behavior.csv}.
    \item Adecuado para grafos: conecta clientes, productos y categorías con relaciones de compra.
  \end{itemize}
\end{frame}

\begin{frame}{Diccionario de datos (resumen)}
  \begin{itemize}
    \item \textbf{Customer}: \texttt{customerId}, edad, género, ubicación, suscripción, compras previas, frecuencia.
    \item \textbf{Product}: \texttt{productId}, nombre, talla, color, temporada, rating promedio.
    \item \textbf{Category}: nombre de categoría.
    \item \textbf{BOUGHT} (relación): monto, descuento, rating, método de pago, fecha de importación.
  \end{itemize}
\end{frame}

\section{Modelado en Neo4J}

\begin{frame}{Esquema de grafo}
  \begin{itemize}
    \item Nodos: \textbf{Customer}, \textbf{Product}, \textbf{Category}.
    \item Relaciones: \textbf{BOUGHT} (cliente $\rightarrow$ producto), \textbf{BELONGS\_TO} (producto $\rightarrow$ categoría).
    \item Restricciones de unicidad: \texttt{customerId}, \texttt{productId}, \texttt{Category.name}.
  \end{itemize}
  \begin{center}
    \includegraphics[width=\linewidth,height=0.72\textheight,keepaspectratio]{../docs/modelo_grafo.png}
  \end{center}
\end{frame}

\begin{frame}{Subgrafo con datos reales}
  \begin{center}
    \includegraphics[width=\linewidth,height=0.72\textheight,keepaspectratio]{../docs/subgrafo_datos_reales.png}
  \end{center}
  \begin{itemize}
    \item Ejemplo de clientes, productos y categorías conectados tras la importación.
  \end{itemize}
\end{frame}

\section{Importación y herramientas}

\begin{frame}{Proceso de importación}
  \begin{enumerate}
    \item Copia del CSV a la carpeta \texttt{import} de Neo4J.
    \item Ejecución de \texttt{constraints.cypher}.
    \item Creación de nodos con \texttt{import\_nodes.cypher}.
    \item Creación de relaciones con \texttt{import\_relationships.cypher}.
  \end{enumerate}
  \vspace{0.5em}
  \textbf{Herramientas:} Neo4J Browser / Desktop, Cypher, Docker Compose, GitHub.
\end{frame}

\section{CRUD y evidencias}

\begin{frame}{Operaciones CRUD (5 por operación)}
  \begin{itemize}
    \item \textbf{CREATE}: clientes, productos, categorías, relaciones de compra.
    \item \textbf{READ}: top productos, clientes por categoría, promedio por método de pago, etc.
    \item \textbf{UPDATE}: actualización de edad, suscripción, ratings promedios, incrementos de compras.
    \item \textbf{DELETE}: eliminación de clientes, relaciones de compra bajas, productos sin compras.
  \end{itemize}
  \begin{center}
    \includegraphics[width=0.75\linewidth]{../docs/ConsultaMayorA50.jpg}
  \end{center}
\end{frame}

\begin{frame}{Evidencia de operaciones}
  \begin{columns}
    \column{0.48\textwidth}
      \includegraphics[width=\linewidth]{../docs/Cliente_nuevo.jpg}
      \vspace{0.3em}
      \includegraphics[width=\linewidth]{../docs/ActualizarEdad.jpg}
    \column{0.48\textwidth}
      \includegraphics[width=\linewidth]{../docs/Cliente9999eliminado.jpg}
  \end{columns}
\end{frame}

\section{Reproducibilidad}

\begin{frame}{Ejecución con Docker}
  \begin{footnotesize}
  \begin{itemize}
    \item Carpeta \texttt{neo4j-docker/} con \texttt{docker-compose.yml}.
    \item Volúmenes: \texttt{data}, \texttt{logs}, \texttt{import}, \texttt{plugins}, \texttt{conf}.
    \item Montaje de \texttt{neo4j/} como \texttt{/scripts} para ejecutar los .cypher directamente.
    \item Comandos principales:
      \begin{itemize}
        \item \texttt{docker compose up -d}
        \item \texttt{cat neo4j/constraints.cypher | docker compose exec -T neo4j cypher-shell -u neo4j -p test123}
      \end{itemize}
  \end{itemize}
  \end{footnotesize}
\end{frame}

\section{Conclusiones}

\begin{frame}{Conclusiones}
  \begin{itemize}
    \item El modelo de grafos captura de forma clara las relaciones cliente--producto--categoría y soporta consultas analíticas.
    \item Constraints e índices garantizan integridad y buen rendimiento en lecturas/actualizaciones.
    \item Las evidencias muestran que las operaciones CRUD funcionan con el dataset real.
    \item El setup Docker asegura reproducibilidad para cualquier integrante del equipo.
    \item El dataset es una buena base para extensiones: recomendaciones, segmentación y análisis estacional.
  \end{itemize}
\end{frame}

\begin{frame}
  \centering
  \Huge ¡Gracias!
\end{frame}

\end{document}
